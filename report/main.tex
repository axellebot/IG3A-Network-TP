%%%%%%%%%%%%%%%%%%%%%%%%%%%%%%%%%%%%%%%%%%%%%%%%%%%%%%%%%%%%%%%%%%%%%%%%%%%%%%%%
% The Legrand Orange Book
% LaTeX Template
% Version 2.2 (30/3/17)
%
% This template has been downloaded from:
% http://www.LaTeXTemplates.com
%
% Original author:
% Mathias Legrand (legrand.mathias@gmail.com) with modifications by:
% Vel (vel@latextemplates.com)
%
% License:
% CC BY-NC-SA 3.0 (http://creativecommons.org/licenses/by-nc-sa/3.0/)
%
% Compiling this template:
% This template uses biber for its bibliography and makeindex for its index.
% When you first open the template, compile it from the command line with the
% commands below to make sure your LaTeX distribution is configured correctly:
%
% 1) pdflatex main
% 2) makeindex main.idx -s StyleInd.ist
% 3) biber main
% 4) pdflatex main x 2
%
% After this, when you wish to update the bibliography/index use the appropriate
% command above and make sure to compile with pdflatex several times
% afterwards to propagate your changes to the document.
%
% This template also uses a number of packages which may need to be
% updated to the newest versions for the template to compile. It is strongly
% recommended you update your LaTeX distribution if you have any
% compilation errors.
%
% Important note:
% Chapter heading images should have a 2:1 width:height ratio,
% e.g. 920px width and 460px height.
%
%%%%%%%%%%%%%%%%%%%%%%%%%%%%%%%%%%%%%%%%%%%%%%%%%%%%%%%%%%%%%%%%%%%%%%%%%%%%%%%%

%----------------------------------------------------------------------------------------
%	DOCUMENT CONFIGURATIONS
%----------------------------------------------------------------------------------------

\documentclass[12pt,a4paper]{book} % Default font size and left-justified equations

\input{structure} % Insert the stucture.tex file which contains the majority of the structure behind the template

\begin{document}

%----------------------------------------------------------------------------------------

%----------------------------------------------------------------------------------------
%	TITLE PAGE
%----------------------------------------------------------------------------------------

\begingroup
\thispagestyle{empty}
\begin{tikzpicture}[remember picture,overlay]
\node[inner sep=0pt] (background) at (current page.center) {\includegraphics[width=\paperwidth]{./Pictures/background.png}};
\draw (current page.center) node [fill=ocre!30!white,fill opacity=0.6,text opacity=1,inner sep=1cm]{\Huge\centering\bfseries\sffamily\parbox[c][][t]{\paperwidth}{\centering Réseau\\[15pt] % Book title
{\Large Compte Rendu de Travaux Pratiques}\\[20pt] % Subtitle
{\huge Axel LE BOT | Andrew LENC}}}; % Author name
\end{tikzpicture}
\vfill
\endgroup
%----------------------------------------------------------------------------------------
%	COPYRIGHT PAGE
%----------------------------------------------------------------------------------------

\newpage
~\vfill
\thispagestyle{empty}

\noindent Copyright \copyright\ 2017-2018 Axel LE BOT \\ % Copyright notice

\noindent Licensed under the Creative Commons Attribution-NonCommercial 3.0 Unported License (the ``License''). You may not use this file except in compliance with the License. You may obtain a copy of the License at \url{http://creativecommons.org/licenses/by-nc/3.0}. Unless required by applicable law or agreed to in writing, software distributed under the License is distributed on an \textsc{``as is'' basis, without warranties or conditions of any kind}, either express or implied. See the License for the specific language governing permissions and limitations under the License.\\ % License information

%----------------------------------------------------------------------------------------
%	INTRODUCTION
%----------------------------------------------------------------------------------------

\pagestyle{empty} % No headers

\chapterimage{./Pictures/cover-start}

\chapter*{Introduction}\addcontentsline{toc}{part}{\texorpdfstring{\protect\@myparttocformat{Introduction}}{Introduction}}

L'origine d'Internet vient des recherche effectué par le gouvernement des États-Unis dans les années 60 afin de construire une communication robuste, tolérante entre ordinateurs.

Le plus souvent Internet est souvent vu comme un nuage. Nous mettons nos donnés à un endroit et puis nous les retrouvons à un autre si nous le voulons, entre les deux il se passe des choses dans le brouillard.

En réalité internet est une inter-connexion entre différents appareils formant un réseau.
Internet est basé sur le modèle TCP/IP lui même découlant du modèle OSI. Le modèle OSI établit des règles permettant aux appareils d'un réseau à communiquer. Il spécifie comment les données doivent être formaté, addressé, envoyé, dirigé et délivré au bon destinataire.

En pratique les réseaux utilisent différents appareils permettant de gérer les communication (Routeur, Commutateur, Serveur, ...), nous allons, durant ces travaux pratique, apprendre à utiliser et configurer ses différents appareils et les différents logiciels afin de créer un réseau.

\cleardoublepage % Forces the table of contents chapter to start on an odd page so it's on the right

\pagestyle{fancy} % Print headers again

%----------------------------------------------------------------------------------------
%	TABLE OF CONTENTS
%----------------------------------------------------------------------------------------

%\usechapterimagefalse % If you don't want to include a chapter image, use this to toggle images off - it can be enabled later with \usechapterimagetrue

\chapterimage{./Pictures/cover-table_of_contents} % Table of contents heading image

\pagestyle{empty} % No headers

\tableofcontents % Print the table of contents itself

\cleardoublepage % Forces the first chapter to start on an odd page so it's on the right

\pagestyle{fancy} % Print headers again

%----------------------------------------------------------------------------------------
%	TABLE OF FIGURES
%----------------------------------------------------------------------------------------

\chapterimage{./Pictures/cover-table_of_contents} % Table of contents heading image

\pagestyle{empty} % No headers

\listoffigures % Print the table of contents itself

\cleardoublepage % Forces the first chapter to start on an odd page so it's on the right

\pagestyle{fancy} % Print headers again

%----------------------------------------------------------------------------------------
%	PART ONE
%----------------------------------------------------------------------------------------

\part{TP}

\chapterimage{./Pictures/cover-pipe}
\chapter{TP1: Installation et découverte}

\section{Installation et prise en main}
\subsection{Installation du système}
À l'aide de notre live USB nous avons pu installer sur notre HDD externe la distribution de linux appelé "Ubunutu".

\subsection{Découverte du système}
Afin d'accéder à des programmes spécifique ou encore lire et moidifier des fichier system nous avons besoin de la commande \mintinline{shell}{sudo} signifiant \texttt{Super User Do}.

\subsection{Installation des outils nécessaires}
Afin d'avoir les outils des base installer sur la machine nous installons les outils nécessaires à l'aide de la commande \mintinline{shell}{sudo apt install wireshark ethtool iperf openssh-server}.

\section{Découverte de l'environnement}
\subsection{L'interface réseau}
À l'aide de la commande \mintinline{shell}{ifconfig -a} nous pouvons analyser les différentes interfaces réseau.\\
Nous pouvons voir que l'interface appelé \textit{eth0}, n'éxiste plus.

\begin{figure}[H]
\centering
\includegraphics[width=400pt]{./TP1/Pictures/ifconfig}
\caption{Ifconfig}
\label{Ifconfig}
\end{figure}

\begin{itemize}
\item Afin de changer d'address IP nous utiliserons la commande suivante :

\begin{minted}{shell}
ifconfig <INTERFACE> <IP_ADDRESS>
\end{minted}
ainsi la commande \mintinline{shell}{ifconfig enp7s4 198.245.10.0} nous permet de mettre l'address \texttt{198.245.10.0} sur l'interface \texttt{enp7s4}.

\item Afin de changer l'adresse MAC nous utiliserons la commande suivante :
\begin{minted}{shell}
ifconfig <INTERFACE> hw ether <MAC_ADDRESS>
\end{minted}
\item Pour changer l'état d'une interface réseau nous utiliserons la commande :
\begin{minted}{shell}
ifconfig <INTERFACE> [down|up]
\end{minted}
\item Pour afficher l'etat des liens IP nous utiliserons la commande :
\begin{minted}{shell}
ip link show
\end{minted}
\item Pour afficher plus d'informations sur les interfaces réseaux :
\begin{minted}{shell}
ethtool <INTERFACE>
\end{minted}

\end{itemize}

\subsection{État du réseau et connexions actives}
Nous pinguons le site : \texttt{google.fr} à l'aide de la commande :
\begin{minted}{shell}
ping <IP_ADDRESS|URL>
\end{minted}

ainsi la commande \mintinline{shell}{ping google.fr} nous renvoie le texte suivant :
\begin{minted}{shell}
64 bytes from 8.8.8.8: icmp_seq=56 ttl=41 time=19.1 ms
64 bytes from 8.8.8.8: icmp_seq=57 ttl=41 time=19.1 ms
\end{minted}
Nous allons ci-dessous présenter les différents utilisation de la commande \mintinline{shell}{netstat} :
\begin{itemize}
  \item  \mintinline{shell}{netstat} : affiche les connexion réseaux
  \item  \mintinline{shell}{netstat -n} : affiche les adresse au format numerique
  \item  \mintinline{shell}{netstat -a} : affiche tout les socket
  \item  \mintinline{shell}{netstat  --tcp} : définit le protocole à lister (ici TCP)
  \item  \mintinline{shell}{netstat --inet --udp} : définit le protocole a lister (ici udp) aficher selon la methode inet
  \item  \mintinline{shell}{netstat -r --inet -6} : affiche la table de routage selon la methode inet avec ipv6
\end{itemize}

\subsection{La politique de routage}
Afin d'afficher les routes, nous utiliserons la commande \mintinline{shell}{route -n}

\begin{figure}[H]
\centering
\includegraphics[width=400pt]{./TP1/Pictures/route}
\caption{Route}
\label{Route}
\end{figure}

La ligne 3 correspond au routage du traffic entre les machines de la salle de TP.\\
Une \texttt{passerelle} (\texttt{gateway}) permet d'accéder à un autre réseaux.\\
Si on souhaite envoyer un message à l'adresse 80.80.80.80 on devra utiliser la passerelle par défaut de la ligne 1, via l'interface \texttt{enp63s0}.\\
La passerelle par défaut sur notre machine est la passerelle \texttt{vl33-miraa-ltp-} à l'adresse IP \texttt{172.24.0.1}.

\subsection{Résolution des noms de domaine}
Afin de résoudre un nom domaine, nous utiliserons la commade suivante :
\begin{minted}{shell}
nslookup <URL>
\end{minted}
Ainsi la commande \mintinline{shell}{nslookup www.yahoo.fr} nous renvoie la ŕeponse suivante :
\begin{minted}{bash}
Server:		172.17.0.5
Address:	172.17.0.5#53

Non-authoritative answer:
www.yahoo.fr	canonical name = rc.yahoo.com.
rc.yahoo.com	canonical name = src.g03.yahoodns.net.
Name:	src.g03.yahoodns.net
Address: 212.82.100.150
\end{minted}

Cette réponse venant du serveur DNS à l'adresse IP \texttt{172.17.0.5} nous permet de connaitre l'adresse IP du serveur \texttt{www.yahoo.fr} : 212.82.100.150\\
Le fichier \textit{/etc/resolv.conf} contient les informations de serveur DNS.\\
En ajoutant les correspondance adresse IP <-> nom de domaine au fichier \textit{/etc/hosts} nous pouvons accéder directement aux machines sans passer pas un serveur DNS.

\section{Manipulations avancées}
\subsection{Étude d'une trame ethernet}

Lors de l'utilisation de la commande \mintinline{shell}{ping}, nous observons que la réponse de la commande affiche un champ \textit{TTL}. TTL signifie Time to live, il indique le nombre de saut réstant avant destination. Si le TTL est trop faible, il sera afficher \textit{Time to live exceeded}.\\
À l'aide de Wireshark nous pouvons sniffer le réseau et observer que le protocole utlisé pour la commande \mintinline{shell}{ping} est \textit{ICMP}.
\begin{figure}[H]
\centering
\includegraphics[width=400pt]{./TP1/Pictures/wireshark}
\caption{Wireshark}
\label{Wireshark}
\end{figure}

\subsection{La commande \mintinline{shell}{traceroute}}

La commande \mintinline{shell}{traceroute <IP_ADDRESS>} permet de tracer un paquet depuis un réseau IP vers un hôte donné.\\
En utilisant \mintinline{shell}{traceroute google.fr} nous obtenons la réponse ci-dessous :

\begin{figure}[H]
\centering
\includegraphics[width=400pt]{./TP1/Pictures/traceroute}
\caption{Traceroute}
\label{Traceroute}
\end{figure}

Les lignes comprenant \texttt{* * *} signifie que le saut sur ce router n'a pas généré de response \texttt{Time-to-live exceeded}\\
On identifie à la première ligne de la réponse l'adresse IP de la passerelle par défaut, ici : \texttt{124.24.0.1}. En effet le premier routeur principale est la passerelle par défaut puisque l'hôte de destination ne fait pas partie du même réseau.

\chapterimage{./Pictures/cover-table_of_contents}
\chapter{TP2}

\chapterimage{./Pictures/cover-table_of_contents}
\chapter{TP3}

%\input{./TP4/tp4.tex} % Skip
\chapterimage{./Pictures/cover-cloud}
\chapter{TP5: Iptables}

\section{Netfilter}
La commande \mintinline{shell}{iptables -P <CHAIN> <POLICY>} nous permet de définir la politique par défaut d' une chaine iptables.

On propose ci-dessous une politique par défaut pour :
\begin{itemize}
  \item Une machine de bureau
  \begin{minted}{shell}
    iptables -P FORWARD DROP
    iptables -P INPUT ACCEPT
    iptables -P OUTPUT ACCEPT
  \end{minted}
  \item Un routeur
  \begin{minted}{shell}
    iptables -P FORWARD ACCEPT
    iptables -P INPUT ACCEPT
    iptables -P OUTPUT ACCEPT
  \end{minted}
\end{itemize}

L'options \mintinline{shell}{--match} ou \mintinline{shell}{-m} permet de faire correspondre d'autre éléments, comme l'adresse IP de la source, le port.

Ci-dessous des exemple de politique dans différent contexte :
\begin{itemize}
  \item Pour autoriser les paquets ayant pour port sources le port 80 à entrer sur un machine nous utiliserons la commande suivante : \mintinline{shell}{iptables -A INPUT -sport 80 --j accept}
  \item Pour autoriser les paquets ayant pour port sources le port 80 à traverser une machine nous utiliserons la commande suivante : \mintinline{shell}{iptables -A FOWARD -sport 80 --j accept}
  \item Pour autoriser les paquets ayant pour port sources le port 53 et ayant pour adresse source \textit{193.50.50.2} à entrer sur une machine nous utiliserons la commande suivante : \mintinline{shell}{iptables -A INPUT -sport 53 -s 193.50.50.2 --j accept}
  \item Pour autoriser les paquets ayant pour port sources le port 53 et ayant pour adresse source \textit{193.50.50.2} à entrer sur une machine nous utiliserons la commande suivante : \mintinline{shell}{iptables -A FORWARD -sport 53 -s 193.50.50.2 --j accept}
  \item Afin d'autoriser en entré que les paquets relatifs à une connexion déjà existante nous utiliserons les commandes suivantes :
  \begin{minted}{shell}
    iptables -P INPUT DROP
    iptables -A INPUT -m established -cstate related --j accept
  \end{minted}
  \item Afin d'autoriser en traversée que les paquets relatifs à une connexion déjà existante nous utiliserons les commandes suivantes :
  \begin{minted}{shell}
    iptables -P FORWARD DROP
    iptables -A FORWARD -m established -cstate related --j accept
  \end{minted}
\end{itemize}

Nous pouvons utilisé la commande \mintinline{shell}{netstats} pour connaitre les ports utilisés.

\section{Manipulation iptables}
Nous pouvons utiliser la commande \mintinline{shell}{iptables -P INPUT DROP} pour bloquer les communication non autorisés.

Nous utiliserons les commandes ci-dessous dans différents contexte :
\begin{itemize}
  \item Pour visualiser les pages web des autres serveurs depuis :
  \begin{itemize}
    \item le serveur : \mintinline{shell}{iptables -A FORWARD -sport 80 --j accept}
    \item le client : \mintinline{shell}{iptables -A INPUT -sport 80 -s 192.168.255.0/24 --j accept}
  \end{itemize}
  \item Pour visualiser la page d'acceuil de google (216.58.208.206) depuis le serveur et le client : \mintinline{shell}{iptables -A INPUT -s 216.58.208.206 --j accept}
  \item Pour visualiser la page d'acceuil de google en utilisant son nom de domaine il faut :
  \begin{itemize}
    \item autoriser le serveur DNS :
    \begin{minted}{shell}
    iptables -A INPUT -sport 53 -s 193.50.50.2 --j accept
    iptables -A INPUT -sport 53 -s 193.50.50.6 --j accept
    \end{minted}
    \item Accepter google.com : \mintinline{shell}{iptables -A INPUT -s google.com --j accept}
    \item Sur le serveur : \mintinline{shell}{iptables -A FORWARD -sport 53 -s 193.50.50.2 --j accept}
  \end{itemize}
\end{itemize}

On enregistre les scripts suivant pour facilité le déploiement :
\begin{itemize}
  \item Pour le server :
  \inputminted{bash}{../sources/TP5/tp5-ex2-server.sh}
  \item Pour le client :
  \inputminted{bash}{../sources/TP5/tp5-ex2-client.sh}
  \item Pour réinitialiser :
  \inputminted{bash}{../sources/TP5/tp5-ex2-reset.sh}
\end{itemize}

Le fichier \textit{/etc/services} sert à faire correspondre chaque port de la machine avec le programme qui l'utilise. Par exemple le port 80 est utilisé par les serveurs HTTP et le port 22 est utilisé par le SSH.

\section{Exercice 3}

L'option \mintinline{shell}{-m limit --limit <PACKETS_COUNT>/<TIME>} permet de limiter le nombre de paquet. Il suffit de l'associer à \mintinline{shell}{iptables} pour quelle prenne effet.\\
Par exemple nous pouvons limiter le nombre de paquets à 5 par heure  \mintinline{shell}{-m limit --limit 5/h}

\paragraph{4.}

\inputminted[linenos]{bash}{../sources/TP5/tp5-ex3-5.sh}

\paragraph{6.}
Le port knocking consiste à envoyer des requêtes sur certains ports dans le bon ordre afin de modifier un pare-feu distant depuis l'extérieur.

\inputminted[linenos]{bash}{../sources/TP5/tp5-ex3-7.sh}

\section{Translation d'adresses}


%----------------------------------------------------------------------------------------
%	CONCLUSION
%----------------------------------------------------------------------------------------

\cleardoublepage % Forces the conclusion chapter to start on an odd page so it's on the right

\pagestyle{empty}

\chapterimage{./Pictures/cover-end}

\chapter*{Conclusion}\addcontentsline{toc}{part}{\texorpdfstring{\protect\@myparttocformat{Conclusion}}{Conclusion}}

Ces TP ont donc démystifiés une grande partie du fonctionnement des réseaux. Bien que l'étude théorique permette d'en comprendre plus, l'application pratique permet d'éclairer les manipulations, et d'en savoir plus sur la réalité de la chose. L'utilisation de ces enseignements peut se faire tout les jours, et permet de mieux savoir diagnostiquer des problèmes de connectivité qui pourrait survenir.

Nous pouvons donc mieux appréhender le rôle des différents composants d'un réseau connecté à Internet, comme les DHCP, et les DNS mais aussi ce qui se cache pour distribuer du contenu, comme les serveurs Apache.

La réalisation de ces TP a donc été enrichissante, et l'enseignant présent a été très utile pour nous aider à leur accomplissement.

\pagestyle{fancy}

%----------------------------------------------------------------------------------------

\end{document}
